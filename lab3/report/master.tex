\documentclass[11pt,a4paper]{article}

% For å kunne skrive norske tegn.
\usepackage[utf8]{inputenc}

% For å inkludere figurer.
\usepackage{graphicx}

% Ekstra matematikkfunksjoner.
\usepackage{amsmath,amssymb}

\usepackage{fancyvrb}

\newenvironment{code}
  {
    \VerbatimEnvironment
    \vskip\baselineskip\hrule
    \begin{Verbatim}[xleftmargin=7pt]%
  }
  {\end{Verbatim}\hrule\vskip\baselineskip}

% Må inkluderes for blant annet å få tilgang til kommandoen \SI (korrekte måltall med enheter)
\usepackage{siunitx}

  % Prikk som multiplikasjonstegn (i steden for kryss).
  \sisetup{exponent-product = \cdot}

  % Pluss-minus-form på usikkerhet (i steden for parentes).
  \sisetup{separate-uncertainty = true}

% For å få tilgang til finere linjer (til bruk i tabeller og slikt).
\usepackage{booktabs}

% For justering av figurtekst og tabelltekst.
\usepackage[font=small,labelfont=bf]{caption}

% Disse kommandoene kan gjøre det enklere for LaTeX å plassere figurer og tabeller der du ønsker.
\setcounter{totalnumber}{5}
\renewcommand{\bottomfraction}{0.95}
\renewcommand{\floatpagefraction}{0.35}

\begin{document}

  % Rapportens tittel:
  \title{Lab 3: Semantics \\ \large{TDT4275: Natural Language Interfaces}}
  \author{Jonas Myrlund}

  % Her ber vi LaTeX om å lage tittelen (til nå har vi bare sagt hva den skal inneholde):
  \maketitle

  \section{Written assignments} % (fold)
  \label{sec:written_assignments}

    \subsection{Feature-based grammars} % (fold)
    \label{sub:feature_based_grammars}

      Documentation exists alongside code. Please see \texttt{feat1.fcfg} for details.

      \subsubsection{Example runs} % (fold)
      \label{ssub:example_runs}

        The packaged tool is simple to use, and to try out new sentences with.
        From the root folder, run \texttt{python run.py --help} for an explanation.

        The default output of the command outputs the following:

        \begin{code}
          I want to spend lots of money            OK
          me want to spend lots of money           FAIL

          tell me about Chez Parnisse              OK
          tell I about Chez Parnisse               FAIL

          I would like to take her out to dinner   OK
          I would like to take she out to dinner   FAIL

          she does not like Italian                OK
          her does not like Italian                FAIL

          this dog runs                            OK
          I runs                                   FAIL
          these dogs runs                          FAIL
        \end{code}

        To run a specific sentence through the parser, the command line flag \texttt{-s/--sentence} is utilized.
        To display detailed trace information and generated parse trees, the \texttt{--debug} flag can be used.

        The command \texttt{python run.py --sentence ``I want to spend lots of money'' --debug} thus yields:

        {\footnotesize
          \begin{code}
  |.I.w.t.s.l.o.m.|
  Leaf Init Rule:
  |[-] . . . . . .| [0:1] 'I'
  |. [-] . . . . .| [1:2] 'want'
  |. . [-] . . . .| [2:3] 'to'
  |. . . [-] . . .| [3:4] 'spend'
  |. . . . [-] . .| [4:5] 'lots'
  |. . . . . [-] .| [5:6] 'of'
  |. . . . . . [-]| [6:7] 'money'
  Feature Bottom Up Predict Combine Rule:
  |[-] . . . . . .| [0:1] Pro[Form='sub', Num='sg', Per=1] -> 'I' *
  Feature Bottom Up Predict Combine Rule:
  |[-> . . . . . .| [0:1] S[] -> Pro[Form='sub', Num=?n] * VP[Num=?n,
  Feature Bottom Up Predict Combine Rule:
  |. [-] . . . . .| [1:2] V[Tense='inf', Type='trans'] -> 'want' *
  Feature Bottom Up Predict Combine Rule:
  |. [-> . . . . .| [1:2] S[] -> V[Tense='inf'] * Pro[Form='obj'] PP[]
  |. [-> . . . . .| [1:2] VP[Num=?n, Tense=?t] -> V[Num=?n, Tense=?t,
  |. [-> . . . . .| [1:2] VP[Num=?n, Tense=?t] -> V[Num=?n, Tense=?t,
  |. [-] . . . . .| [1:2] V[Num='pl', Tense='pres'] -> V[Tense='inf']
  Feature Bottom Up Predict Combine Rule:
  |. [-] . . . . .| [1:2] VP[Num='pl', Per=?p, Tense='pres'] -> V[Num=
  |. [-> . . . . .| [1:2] VP[Num=?n, Tense=?t] -> V[Num=?n, Tense=?t,
  |. [-> . . . . .| [1:2] VP[Num=?n, Tense=?t] -> V[Num=?n, Tense=?t,
  |. [-> . . . . .| [1:2] VP[Num=?n, Tense=?t] -> V[Num=?n, Tense=?t,
  Feature Bottom Up Predict Combine Rule:
  |. . [-] . . . .| [2:3] Inf[] -> 'to' *
  Feature Bottom Up Predict Combine Rule:
  |. . [-> . . . .| [2:3] VP[Tense='inf', +inf] -> Inf[] * VP[Tense='i
  |. . [-] . . . .| [2:3] Aux[] -> Inf[] *
  Feature Bottom Up Predict Combine Rule:
  |. . [-> . . . .| [2:3] VP[+aux] -> Aux[] * VP[Tense='inf'] {}
  |. . [-> . . . .| [2:3] Aux[] -> Aux[] * 'not' {}
  Feature Bottom Up Predict Combine Rule:
  |. . . [-] . . .| [3:4] V[Tense='inf', Type='trans'] -> 'spend' *
  Feature Bottom Up Predict Combine Rule:
  |. . . [-> . . .| [3:4] S[] -> V[Tense='inf'] * Pro[Form='obj'] PP[]
  |. . . [-> . . .| [3:4] VP[Num=?n, Tense=?t] -> V[Num=?n, Tense=?t,
  |. . . [-> . . .| [3:4] VP[Num=?n, Tense=?t] -> V[Num=?n, Tense=?t,
  |. . . [-] . . .| [3:4] V[Num='pl', Tense='pres'] -> V[Tense='inf']
  Feature Bottom Up Predict Combine Rule:
  |. . . [-] . . .| [3:4] VP[Num='pl', Per=?p, Tense='pres'] -> V[Num=
  |. . . [-> . . .| [3:4] VP[Num=?n, Tense=?t] -> V[Num=?n, Tense=?t,
  |. . . [-> . . .| [3:4] VP[Num=?n, Tense=?t] -> V[Num=?n, Tense=?t,
  |. . . [-> . . .| [3:4] VP[Num=?n, Tense=?t] -> V[Num=?n, Tense=?t,
  Feature Bottom Up Predict Combine Rule:
  |. . . . [-> . .| [4:5] N[Num='mass'] -> 'lots' * 'of' 'money' {}
  Feature Single Edge Fundamental Rule:
  |. . . . [---> .| [4:6] N[Num='mass'] -> 'lots' 'of' * 'money' {}
  Feature Single Edge Fundamental Rule:
  |. . . . [-----]| [4:7] N[Num='mass'] -> 'lots' 'of' 'money' *
  Feature Bottom Up Predict Combine Rule:
  |. . . . [-----]| [4:7] NP[Num='mass'] -> N[Num='mass'] *
  Feature Bottom Up Predict Combine Rule:
  |. . . . [----->| [4:7] S[] -> NP[Num=?n] * VP[Num=?n, -inf] {?n: 'm
  Feature Single Edge Fundamental Rule:
  |. . . [-------]| [3:7] VP[Num=?n, Tense='inf'] -> V[Num=?n, Tense='
  |. . . [-------]| [3:7] VP[Num='pl', Tense='pres'] -> V[Num='pl', Te
  Feature Single Edge Fundamental Rule:
  |. . [---------]| [2:7] VP[Tense='inf', +inf] -> Inf[] VP[Tense='inf
  |. . [---------]| [2:7] VP[+aux] -> Aux[] VP[Tense='inf'] *
  Feature Single Edge Fundamental Rule:
  |. [-----------]| [1:7] VP[Num=?n, Tense='inf'] -> V[Num=?n, Tense='
  |. [-----------]| [1:7] VP[Num='pl', Tense='pres'] -> V[Num='pl', Te
  Feature Single Edge Fundamental Rule:
  |[=============]| [0:7] S[] -> Pro[Form='sub', Num='sg'] VP[Num='sg'
  Feature Single Edge Fundamental Rule:
  |. [-----------]| [1:7] VP[Num=?n, Tense='inf'] -> V[Num=?n, Tense='
  |. [-----------]| [1:7] VP[Num='pl', Tense='pres'] -> V[Num='pl', Te
  I want to spend lots of money            OK

  (S[]
    (Pro[Form='sub', Num='sg', Per=1] I)
    (VP[Num=?n, Tense='inf']
      (V[Tense='inf', Type='trans'] want)
      (VP[Tense='inf', +inf]
        (Inf[] to)
        (VP[Num=?n, Tense='inf']
          (V[Tense='inf', Type='trans'] spend)
          (NP[Num='mass'] (N[Num='mass'] lots of money))))))

  (S[]
    (Pro[Form='sub', Num='sg', Per=1] I)
    (VP[Num=?n, Tense='inf']
      (V[Tense='inf', Type='trans'] want)
      (VP[+aux]
        (Aux[] (Inf[] to))
        (VP[Num=?n, Tense='inf']
          (V[Tense='inf', Type='trans'] spend)
          (NP[Num='mass'] (N[Num='mass'] lots of money))))))
\end{code}

        }

      % subsubsection example_runs (end)

    % subsection feature_based_grammars (end)

    \subsection{First Order Logic} % (fold)
    \label{sub:first_order_logic}

      FOL-expressions for sentences: \\

      \begin{tabular}{ll}
        Sharks do not eat birds & $ \forall x, y \: ( Shark(x) \land Bird(y) \land \neg Eats(x, y) ) $ \\
        Not all birds lay eggs & $ \neg ( \forall x \: ( Bird(x) \land LaysEggs(x) ) $ \\
      \end{tabular}

      \subsubsection{NLTK-format} % (fold)
      \label{ssub:nltk_format}

        \begin{tabular}{ll}
          Sharks do not eat birds & \texttt{all x y.(Shark(x) \& Bird(y) \& -Eats(x, y))} \\
          Not all birds lay eggs  & \texttt{-(all x.(Bird(x) \& LaysEggs(x)))} \\
        \end{tabular}

      % subsubsection nltk_format (end)

      \subsubsection{Running the expressions} % (fold)
      \label{ssub:running_the_expressions}

        The expressions run nicely through the NLTK Logic Parser.
        Calling \texttt{free()} on the resulting objects yields empty sets, as expected.

        \emph{Free variables}, contrary to \emph{bound variables},
        means that the variable is not associated with a quantifier, such as $\forall$ or $\exists$.
        In the case of the above expressions, all cases of variables are both bound to $\forall$-quantifiers.

      % subsubsection running_the_expressions (end)

      \subsubsection{World models} % (fold)
      \label{ssub:world_models}

        The following code builds a simple set of logical expressions:

        \begin{code}
          lp = nltk.LogicParser()

          a = lp.parse('exists x.(samfundet(x) and school(x))')
          b = lp.parse('smart(jonas)')
          c = lp.parse('-smart(jonas)')
        \end{code}

        To execute them together, we can run them in the following manner:

        \begin{code}
          mace = nltk.Mace()

          mace.build_model(None, [a, b]) # => True
          mace.build_model(None, [a, c]) # => True
          mace.build_model(None, [b, c]) # => False
        \end{code}

      % subsubsection world_models (end)

    % subsection first_order_logic (end)

  % section written_assignments (end)

  \section{Lambda-based semantics} % (fold)
  \label{sec:lambda_based_semantics}

    \begin{equation}
      \forall x \exists y \: Near(x, y) \; \Rightarrow \; Hello
    \end{equation}

  % section lambda_based_semantics (end)

\end{document}